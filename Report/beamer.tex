\documentclass[8pt]{beamer}
\usepackage[french]{babel}
\usepackage[utf8]{inputenc}
\usepackage[T1]{fontenc}
\usepackage[titlenumbered,ruled,noend,french,onelanguage,linesnumbered]{algorithm2e}
\usepackage{tikz}
\usepackage{graphicx}
\usepackage{array}
\usepackage{multirow}
\usepackage{caption}
\usepackage{subcaption}
\usetheme{Antibes}
\usepackage{tabulary}



\setbeamertemplate{footline}[frame number]

\title{Débruitage d'image par l'utilisation d'un perceptron multi-couches}
\author{VERNAY Amélie \\ FATTOUHY Mohamed \\ ESTEVE Nathan \\ NGUYEN Louis}

\titlegraphic{\includegraphics[width=2.5cm,height=2.5cm]{../datasets/images/Logo_FDS.png}}

\AtBeginSection[]
{
  \begin{frame}
    \frametitle{Table of Contents}
    \tableofcontents[currentsection]
  \end{frame}
}


\setbeamertemplate{navigation symbols}{}


\begin{document}


\begin{frame}
\titlepage
\end{frame}


\begin{frame}
\frametitle{Sommaire}
\tableofcontents
\end{frame}


\section{Contexte}

\begin{frame}
\frametitle{Problématique de débruitage}
\end{frame}


\section{Les bruits}

\begin{frame}
% Mettre les images bruitées
\end{frame}

\begin{frame}
% Dessin de MLP + expliqué notion de patches (phase d'entrainement)
\end{frame}


\begin{frame}
% Expliqué INPUT/OUTPUT + POIDS W + LOSS (SNR)
\end{frame}

\begin{frame}
% Présentez résultats images avec AWG noise (sigma=25)
\end{frame}

\begin{frame}
% Présentez résultats images pour autres bruits (faire plusieurs slides)
\end{frame}

\begin{frame}
% Bound
\end{frame}


\begin{frame}
% Block-Matching
\end{frame}






\section*{}

\begin{frame}{Conclusion}
\begin{block}{}
\begin{itemize}
\item {L'estimateur LASSO est plus performant que l'OLS lorsque le pourcentage de variables actives est faible.}
\item {Le LASSO, appliqué à nos données génomiques, ne donne pas toujours des résultats interprétables.}
\item {On a donc mis en place le modèle linéaire multi-réponses et l'estimateur MultiTaskLASSO.}
\item {On a finalement appliqué le MultiTaskLASSO aux données : on a ainsi sélectionné les variables actives communes à toutes les réponses.}
\end{itemize}
\end{block}
\end{frame}


\end{document}