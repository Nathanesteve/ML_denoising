\documentclass[10pt,a4paper]{article}
\usepackage[left=1.5cm,right=1.5cm,top=1.5cm,bottom=1.5cm]{geometry}
\usepackage[utf8]{inputenc}
\usepackage[T1]{fontenc}
\usepackage{lmodern}
\usepackage{mathtools}
\usepackage{amsthm}
\usepackage{amsfonts}
\usepackage{amssymb}
\usepackage{dsfont}
\usepackage{longtable}
\usepackage{float}
\usepackage[justification=centering]{caption}
\usepackage{enumitem} % lists without bullets
%\usepackage{subcaption}
%\usepackage{graphicx}
%\usepackage{fancybox}
%\usepackage[dvipsnames,svgnames]{xcolor}
%\usepackage{nicematrix}
%\usepackage{tikz}
%\usetikzlibrary{fit}
%\usepackage{changepage}
\parindent 0ex
\usepackage[english]{babel}

\usepackage[dvipsnames]{xcolor}

% new commands
\newcommand{\svs}{\vspace{9pt}}
\newcommand{\mvs}{\vspace{27pt}}
\newcommand{\bvs}{\vspace{47pt}}
\newcommand{\ourparagraph}[1]{\paragraph{#1}}

\definecolor{newteal}{RGB}{0,118,155}
\usepackage{footnotebackref}
\usepackage{hyperref}
\hypersetup{
    colorlinks=true,
    linkcolor=Maroon,
    urlcolor=Maroon,
    citecolor=newteal}
%\usepackage{tcolorbox}

% references
\usepackage[backend=biber, citetracker=true, natbib=true]{biblatex}
%\usepackage[backend=biber,style=alphabetic,citetracker=true,natbib=true]{biblatex}
\usepackage[toc,page]{appendix}
\addbibresource{./references.bib}

% header
\author{Esteve Nathan, Fattouhy Mohamed, Nguyen Louis, Vernay Amélie}
\title{%
    \begin{minipage}\linewidth
        \centering
        Image denoising with multi-layer perceptrons
        \vskip3pt
        \large 
        HAX907X - Apprentissage statistique
        %\vskip3pt
        %Report
    \end{minipage}
}


\begin{document}

\maketitle
% Dans l'intro il faut parler du PSNR qui sera notre valeur pour determiner si on a fait un "bon denoising" ?
\section{Introduction}

The article we worked on \citep{denoise} aims to learn the mapping from a noisy image (the image pixels undergo random fluctuations), to a noise-free image directly with plain multi-layer perceptrons (MLP).

\svs

Images are invariably corrupted by some degree of noise, which strength and type depends on the imaging process. Image denoising seeks to find a clean image given only its noisy version.

\svs

Its complexity requires to split the image into possibly overlapping smaller areas, called patches, denoised separately and then placed at the location of their noisy counterparts. However, the size of the patches affect the quality of the denoising function : large patches potentially lead to better result, but the function might be difficult to model.

\svs

Among the numerous existing types of noise, the authors mainly focused on additive white and Gaussian-distributed noise with known variance (AWG noise), but they will show that the method can also be adapted to mixed Poisson-Gaussian noise, JPEG artifacts, salt and pepper noise and noise that resembles stripes.

\svs

%\section{Multi-layer perceptron}

% The calculations with this kind of matrices requires therefore great capacity of calculations. (transition...)

%The computationally most intensive operations in an MLP are %the matrix-vector multiplications. So for their experiments %they used Graphics Processing Units (GPUs) rather than Central %Processing Units (CPUs), because of their ability to %efficiently parallelize operations.

%MLP are very expensive in calculation time during their learning phase. Indeed the calibration of such a network requires a lot of algebraic calculation, more precisely matrix-vector products, which are the computationally most intensive operations.
%For the experiments, Graphics Processing Units (GPUs) are better than Central Processing Units (CPUs), because of their ability to efficiently parallelize operations.

%Expliquer les MLP et ajouter une phrase qui dit que dans la mesure où ce sont des matrix-verctor product c'est parallélisable et GPU...

% -----------------------------------------------------



\section{Multi-layer perceptron for image denoising}

A multi-layer perceptron is a particular architecture of neural network. In this architecture we have an input layer, an output layer and many hidden layers, each neuron of a hidden layer being connected to every neuron of the previous and the next ones. Signals between neurons are weighted by randomly initialized weights, updated by backpropagation to compute the gradient of a loss function. With MLP the weight matrix might be dense.

\svs

To find a denoising function with MLP, the authors use pairs of noisy image patch (input) and clean image patch (output). To make it efficient, the data is normalized and the weights $w$ sampled from an uniform distribution :  \\%Weight are randomly initialize following a uniform distribution:\\
$$w \sim \left[-\frac{\sqrt{6}}{\sqrt{n_j + n_{j+1}}}, \frac{\sqrt{6}}{\sqrt{n_j + n_{j+1}}}\right]$$ 

\svs
where $n_j$ and $n_{j+1}$ are the number of neurons in the input and output sides of the layer, respectively.

%Those steps ensure that all parts of the activation function, which is the sigmoid function, are reached.
\svs

To update those weights, they use the stochastic gardient descent, applied to our loss function, defined as the mean squared error between $f(x)$ (denoised patch) and $y$ (clean patch), minimizing pixel-wise. With this choise of loss function, we maximise the PSNR . %The number of hiden layers and their size are: (inserer les tailles) Denoising quality is measured with the signal-to-noise ratio (SNR) : SNR = $20 \times \log_{10}(\dfrac{255}{\sqrt{MSE})}$
Denoising quality is measured with the Peak Signal-To-Noise Ratio (PSNR) : PSNR = $20 \times \log_{10}(m/\sqrt{\mathrm{MSE}})$ (dB), where $m$ is the maximum possible pixel value of a given image. It is the ratio between the maximum possible power of a signal (interpretable signal) and the power of corrupting noise (stray signal)\footnote{\href{https://en.wikipedia.org/wiki/Peak\_signal-to-noise\_ratio}{https://en.wikipedia.org/wiki/Peak\_signal-to-noise\_ratio}}.

\svs

% Expliquer "stochastic gradient descent" (voir tp charlier)
Furthermore, to keep a steady learning rate while modifying the number $N$ of hidden units per layer, they divide it by $N$ in each layer. The basic learning rate was set to 0.1. 

\svs

The number of hidden layers, as well as $N$, determine the capacity of the model. In pratice, it's often better to use a large number of hidden layers with fewer hidden units each.

\svs

All experiments are performed on grey-image images, but the MLPs could also be trained on color images. They used images from six different datasets, and performed no pre-processing but the transform to grey-scale on the training images. 

\svs 

For each type of noise and each noise level, an MLP is trained. This allows us to obtain a
denoising function for each configuration.
Once the MLP is correctly trained, the next step is to compare the results obtained with other
denoising method.

Once the MLP is trained, the next step is to compare the results obtained with other denoising methods.


%To evaluate their approche, they mainly focused on a standard test dataset, $standard\ test\ images$, and AWG noise with $\sigma=25$. However, they show results for others noise levels, other types of noise and other image sets, to compare the performance of different methods.


% Coût/GPU/BM3D

\svs
%After $3.5 \times 10^8$ backpropagations we obtain the denoising fonction that will be use to compare with the other method.

\section{Results and comparison with existing algorithms}

% Faire une sous-partie pour parler des 4 autres algo
% Faire une sous-partie pour AWG avec sigma qui varie.
% Faire une sous-partie avec d'autres bruits.


\ourparagraph{Existing algorithms -}{

Image denoising is a well-known problem, thus denoising methods are numerous and diverse. In order to evaluate the efficiency of the method, they compared the results against the following algorithms :

\svs
% - BM3D is an ingeneered approach that doesn't rely on learning, and a non-local method
- \textbf{BM3D} (2007), this method does not explicitly use an image prior, it use the fact that images contain self-similarities.

- \textbf{NLSC} (2010), a dictionary-based algorithm witch exploit self-similarities in image like BM3D.
%- NLSC is a non-local, dictionary-based algorithm
% non-local ?

Both these methods are considered the state-of-the-art in image denoising.

- \textbf{EPLL} (2011) is a learning-based approach, using a maximisation of likelihood of several patch.

- \textbf{KSVD} (2006) is a dictionary method based on sparse linear combination of dictionary elements.
%- \textbf{KSVD} (2006) is a dictionary-based algorithm that achieves better results than previous state-of-the-art methods
\svs

They chose these algorithms for their comparison because they achieve excellent results, with different approaches.}

\ourparagraph{Comparaison on AWG noise -}{
Let's present the results achieved with an MLP on AWG noise with $\sigma=25$.

\svs 

The MLP $(39 \times 2, 3072, 3072, 2559, 2047, 17 \times 2)$ was trained for approximately $3.5 \times 10^8$ backprops and delivered the best results.

\svs 

Out of the 11 \textit{standard test images}, the MLP approach achieves the best result on 7 images and is the runner-up on one image.
% runner-up ?
Their method is clearly inferior to BM3D and NLSC on both of the images which contain a lot of regular structure. However, it outperforms KSVD on these images, even though KSVD is also an algorithm well-suited for images with regular structure. Furthermore, they also outperform both KSVD and EPLL on every image of the dataset.

\svs

They now compare the MLP method to EPLL, BM3D and NLSC on the five larger test sets : \textit{Berkeley\ BSDS500}, \textit{Pascal VOC 2007}, \textit{Pascal VOC 2011}, \textit{McGill}, and \textit{ImageNet}, with a total of $2500$ test images.

\svs

Their method outperforms EPLL on $99.5\%$ of the $2500$ images, and BM3D on $92\%$ of it. It also outperforms NLSC on $80\%$ of the test sets ; the initial dictionary of NLSC was trained on a subset of \textit{Pascal VOC 2007}, which explains its good results.}

\ourparagraph{Comparison on different noise variances -}{
Now we present the results obtained by their approach on four other
% marre du they faudrait trouver une strat
noise levels : $\sigma = 10$ (low noise), $\sigma = 50$
(high noise), $\sigma = 75$ (very high noise) and $\sigma = 170$ (extremely high noise). 

\svs

On $2500$ test images, the resulst show that MLP is not very efficient for low level noise ($\sigma=10$) compared to BM3D. Indeed, MLP outperform BM3D on $1876$ of the $2500$ images ($75.04\%$).
For ($\sigma = 50$), MLP outperform BM3D on $2394$ ($95.76 \%$) of the $2500$ images, and for ($\sigma = 75$) on $2440$ ($97.60\%$) of the $2500$ images.

\svs 

To know how MLP behaves on the noise levels they have been trained on, they varied $\sigma$ between $5$ and $100$ in steps of $5$.
Results shows that MLPs achieve better results than BM3D on the noise levels they have been trained on.

\svs
However, for noise levels they have not been
trained on, the MLPs performance degrades quickly.

Exceptions are the MLPs trained on $\sigma = 50$ and $\sigma = 75$, which also outperform BM3D on $\sigma = 45$ and $\sigma = 55$ (MLP was trained on $\sigma = 50$) and $\sigma = 70$ and $\sigma = 80$ (MLP trained on $\sigma = 75$).

\svs

So the conclusion is the following : their approach is particularly well suited for medium to high noise
levels ($\sigma \geq 50$). MLPs outperform the previous state-of-the-art on all noise levels, but for $\sigma = 10$. However, MLPs has to be trained on each noise level in order to achieve good results.}

\svs

\ourparagraph{Comparison on different type of noise -}{

%They compared BM3D with MLP for strip noise. After training on more than 82 million examples, MLP outperforms BM3D for all other type of noise.
%For  Salt and Pepper noise, they compare BM3D, MLP and a new method: Median filtering (a common algorithm for removing Salt and Pepper noise). BM3D and Median filtering give worse results than MLP.
%For JPEG-artifact they compare MLP with two methods, JPEG-de-Bloking theand the classic method to attenuate this type of noise. (shift the images, re-apply JPEG compression, shift back and average) An MLP trained on 58 million training examples with that noise outperforms both methods.

Their learning method can be use on other type of noise just changing the training data set, by corrupting the image dataset with the type of noise given. For stripe noise, Salt and Pepper noise, MLP outperform BM3D and the other classical methods.

For JPEG quantization artifacts and mixed Poisson-Gaussian noise MLP seems to be competitive with the state-of-the-art.}

\section{Theoritical bound}

\ourparagraph{Clustering-based bounds -}{Actual denoising algorithms, especially BM3D, are very close to an inherent limit on denoising quality (estimated in \citep{deaddenoise}) for images with rich geometric structure. Yet, MLP outperform BM3D by approximatively $0.4$ dB  on this type of images, which is a significant improvement.

\svs

\ourparagraph{Bayesian bounds -}{Levin and Nadler \citep{Levin2011NaturalID} estimated theoretical bounds on how well any denoising algorithm can perform, which depends on the patch size.


is about patch-based method estimate denoising bounds in a Bayesian framework, for a given patch size. Their approach results exceed these bounds, that's beacause they use larger patches than assumed by Levin and Nadler (2011).

Similar bounds estimated for infinite patch sizes are estimated by Levin et al (2012). MLPs make significant progress in achieving these bounds: their approach achieves almost half the theoretically possible gain over BM3D. Levin et al (2012) agree with Chatterjee and Milanfar (2010) that there is little room for improvement on patches with complex textures.}

\section{Block-matching MLPs}

Recent denoising algorithms such as BM3D and NLSC rely on block-matching procedures. The idea is to find the patches most similar to a reference patch.

Knowing that block-matching procedures are effective, let's see if a combination with MLPs can achieve better results, especially images with repeating structures, on which BM3D and NLSC outperform MLPs.

They train MLPs that take as input a reference patch and its nearest neighbors (similar patches).

The block-matching procedure slows down the training by approximatively 2. The MLP they used has four hidden layers (4095,2047,2047,2047). Compared to BM3D, this method always select the same number of neighbors, directly in the noisy image, and has fewer hyper-parameters.

\ourparagraph{Results -}{
The mean result achieved by the block-matching MLP on the $11$ \textit{standard test images} is $0,07$dB higher than the plain MLP. It outperfoms the plain MLP on $7$ images, especially on images with repeating structure. However, BM3D and NLSC still provide better results on this kind of images.

On larger test sets, both methods achieve quite the same results : plain MLPs perform better on images with smooth surfaces, and block-matching MLPs provide better results on images with repeating structure, whithout outperforming BM3D and NLSC.

However, it's noticeable that block-matching MLPs use less information as input than plain MLPs.}

\section{Conclusion}

Through this paper we can improve the MLP. Whether in the improvement to reach the theoretical limits, or the debriefing. Indeed, when the MLP is trained on a specific type of noise, it provides very good results on the same noise. However on some image paternity, or with low variances of other method (BM3D) seems to compete with the MLP.

\printbibliography

\end{document}