\documentclass[10pt,a4paper]{article}
\usepackage[left=2cm,right=2cm,top=2cm,bottom=2cm]{geometry}
\usepackage[utf8]{inputenc}
\usepackage[T1]{fontenc}
\usepackage{lmodern}
\usepackage{mathtools}
\usepackage{amsthm}
\usepackage{amsfonts}
\usepackage{amssymb}
\usepackage{dsfont}
\usepackage{longtable}
\usepackage{float}
\usepackage[justification=centering]{caption}
\usepackage{enumitem} % lists without bullets
%\usepackage{subcaption}
%\usepackage{graphicx}
%\usepackage{fancybox}
%\usepackage[dvipsnames,svgnames]{xcolor}
%\usepackage{nicematrix}
%\usepackage{tikz}
%\usetikzlibrary{fit}
%\usepackage{changepage}
\parindent 0ex
\usepackage[english]{babel}

\usepackage[dvipsnames]{xcolor}

% new commands
\newcommand{\svs}{\vspace{9pt}}
\newcommand{\mvs}{\vspace{27pt}}
\newcommand{\bvs}{\vspace{47pt}}

\DeclarePairedDelimiter{\ceil}{\lceil}{\rceil}




\usepackage{footnotebackref}
\usepackage{hyperref}
\hypersetup{
    colorlinks=true,
    linkcolor=Maroon,
    urlcolor=Maroon}
%\usepackage{tcolorbox}

% references
\usepackage[backend=biber,style=authoryear,citetracker=true,natbib=true]{biblatex}
\usepackage[toc,page]{appendix}
\addbibresource{./references.bib}

% header
\author{Esteve Nathan, Fattouhy Mohamed, Nguyen Louis, Vernay Amélie}
\title{%
    \begin{minipage}\linewidth
        \centering
        Image denoising with multi-layer perceptrons
        \vskip3pt
        \large 
        HAX907X - Apprentissage statistique
        %\vskip3pt
        %Report
    \end{minipage}
}


\begin{document}

\maketitle

\section{Introduction}

This article\cite{TODO} aims to learn the mapping from a noisy image to a noise-free image directly with plain multi-layer perceptrons (MLP), applied to smaller areas, called patches. The denoised image patches are then combined into a denoised image.

\svs

Images are invariably corrupted by some degree of noise, which strength and type depends on the imaging process. Image denoising seeks to find a clean image given only its noisy version.

\svs

Its complexity requires to split the image into possibly overlapping patches, denoised separately.

\svs

However, the size of the patches affect the modelling of the denoising function and the denoising results.

\svs

Among the numerous existing types of noise, we will mainly focus on additive white and Gaussian-distributed noise with known variance (AWG noise), but the method can also adapted to mixed Poisson-Gaussian noise, JPEG artifacts, salt and pepper noise and noise that resembles stripes.


\section{Multi-layer perceptron}

In a perceptron, each neuron of a hidden layer is connected to every neuron of the previous and next layers.

\svs

Expliquer les MLP et ajouter une phrase qui dit que dans la mesure où ce sont des matrix-verctor product c'est parallélisable et GPU...

% -----------------------------------------------------
\begin{thebibliography}{8}

\svs

\bibitem{denoise} 
\href{https://arxiv.org/abs/1211.1544}{Christopher \textsc{Harold}, Christian J. \textsc{Schuler}, Stefan \textsc{Harmeling}. \emph{Image denoising with multi-layer perceptrons, part 1: comparison with existing algorithms and with bounds}, Journal of Machine Learning Research (2012)}


\end{thebibliography}

\end{document}